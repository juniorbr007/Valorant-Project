\documentclass[conference]{IEEEtran}

% --- PACOTES ---
\usepackage[utf8]{inputenc}
\usepackage[T1]{fontenc}
\usepackage{graphicx}   % Inserção de imagens
\usepackage{amsmath}    % Matemática
\usepackage{booktabs}   % Tabelas bonitas
\usepackage{multirow}   % Mesclar linhas
\usepackage{caption}    % Legendas
\usepackage{subcaption} % Subfiguras
\usepackage{float}      % Para [H]
\usepackage{hyperref}   % Links
\usepackage{siunitx}    % Unidades

% --- CAMINHO DAS FIGURAS ---
\graphicspath{{analysis_results/}}

% --- TÍTULO E AUTOR ---
\title{Cypher's Edge: A Statistical Validation of Hybrid Feature Engineering and Classification Models for League of Legends Match Prediction}

\author{
    \IEEEauthorblockN{João Batista de Aguiar Júnior}
    \IEEEauthorblockA{
        Programa de Pós-Graduação em Engenharia da Computação\\
        Universidade de Pernambuco (UPE)\\
        Email: joaobatistajunior.dev@gmail.com
    }
}

\begin{document}

\maketitle

\begin{abstract}
This study presents Cypher's Edge, a hybrid predictive system designed to estimate match outcomes in \textit{League of Legends} using performance data extracted from the Riot Games API. The system integrates structured data extraction, automated dataset generation, feature engineering, and machine learning models trained to classify match results. To validate and compare the performance of different classification approaches, we conducted a comprehensive analysis using cross-validation, confusion matrices, feature importance ranking, ROC curves, and statistical tests such as the Friedman test with the Nemenyi post-hoc procedure. Results demonstrate that ensemble models outperform baseline approaches from existing literature, with specific performance indicators consistently exhibiting high discriminative value.
\end{abstract}

\begin{IEEEkeywords}
League of Legends, Machine Learning, Classification, Feature Engineering, Friedman Test, Nemenyi, ROC Curve, Riot Games API.
\end{IEEEkeywords}

% --- INTRODUÇÃO ---
\section{Introduction}
Online multiplayer games such as \textit{League of Legends} depend heavily on real-time performance metrics to evaluate player skill and match outcomes. Predicting these outcomes has become a relevant subject of study both for gameplay optimization and understanding competitive environments. Previous work often lacks statistical rigor and reproducible pipelines. 

Cypher's Edge automates dataset construction, training, evaluation, and figure generation. Once new match data is collected, all visualizations, tables, and statistical analyses are regenerated automatically through LaTeX, ensuring consistency and scientific traceability.

% --- ARQUITETURA ---
\section{System Architecture}
Fig.~\ref{fig:architecture} shows the architecture of Cypher's Edge.

\begin{figure}[H]
    \centering
    \includegraphics[width=\linewidth]{architecture_diagram.png}
    \caption{Architecture of Cypher's Edge showing the flow between data collection, processing, and evaluation.}
    \label{fig:architecture}
\end{figure}

% --- TRABALHOS RELACIONADOS ---
\section{Related Work}
A baseline model from Pan et al. (2019) serves as reference, using normalized combat, economy, and survivability indicators. Their work validates feature importance but lacks automated statistical analysis and reproducibility.

% --- METODOLOGIA ---
\section{Methodology}

\subsection{Dataset Construction}
Data was collected exclusively from the Riot Games API. Initial experiments used a single player dataset, but the system supports multi-user aggregation.

\subsection{Feature Engineering}
Categorical features were one-hot encoded, numerical features standardized, and irrelevant columns removed. Metrics include KDA, damage, vision, objectives, and gold indicators.

\subsection{Models Evaluated}
\begin{itemize}
    \item Random Forest
    \item MLP Classifier
    \item Gaussian Naïve Bayes
    \item Literature Baseline (Pan et al.)
\end{itemize}

\subsection{Cross-validation}
All models were evaluated using 10-fold CV, storing accuracy, macro F1-score, and ROC curves per fold.

% --- RESULTADOS ---
\section{Results}

\subsection{Comparative Table of Models}
\begin{table}[H]
\centering
\caption{Comparative Results of the Systems}
\label{tab:results}
\begin{tabular}{lccc}
\toprule
System & Classifier & Accuracy & F1 (Macro) \\
\midrule
Cypher's Edge & Random Forest & 77.8\% & 75.0\% \\
(Proposed) & MLP Classifier & 69.4\% & 67.4\% \\
& Naïve Bayes & 65.0\% & 62.7\% \\
Literature Model & Baseline & 58.1\% & 55.4\% \\
\bottomrule
\end{tabular}
\end{table}

\subsection{Feature Importance}
\begin{figure}[H]
    \centering
    \includegraphics[width=\linewidth]{feature_importance.png}
    \caption{Top-15 most relevant features according to Random Forest (Gini Importance).}
    \label{fig:feature_importance}
\end{figure}

\subsection{Confusion Matrices}
\begin{figure}[H]
    \centering
    \begin{subfigure}[b]{0.48\textwidth}
        \includegraphics[width=\linewidth]{confusion_matrix_Cyphers_Edge_-_Random_Forest.png}
        \caption{Random Forest}
    \end{subfigure}
    \hfill
    \begin{subfigure}[b]{0.48\textwidth}
        \includegraphics[width=\linewidth]{confusion_matrix_Cyphers_Edge_-_MLP_Classifier.png}
        \caption{MLP Classifier}
    \end{subfigure}
    \caption{Confusion matrices for evaluated classifiers.}
    \label{fig:confusion_matrices}
\end{figure}

\subsection{Cross-Validation Boxplot}
\begin{figure}[H]
    \centering
    \includegraphics[width=\linewidth]{cv_boxplot.png}
    \caption{Distribution of CV accuracy scores across folds for all classifiers.}
    \label{fig:cv_boxplot}
\end{figure}

\subsection{Ranking of Models}
\begin{figure}[H]
    \centering
    \includegraphics[width=\linewidth]{ranking_plot.png}
    \caption{Average performance ranking of evaluated models. Lower rank indicates better performance.}
    \label{fig:ranking_plot}
\end{figure}

\subsection{ROC Curves}
\begin{figure}[H]
    \centering
    \includegraphics[width=\linewidth]{roc_curves.png}
    \caption{ROC curves for all classifiers, averaged across 10-fold cross-validation.}
    \label{fig:roc_curves}
\end{figure}

\subsection{Critical Difference Diagram}
\begin{figure}[H]
    \centering
    \includegraphics[width=\linewidth]{nemenyi_plot_manual.png}
    \caption{Critical Difference diagram using the Nemenyi post-hoc test.}
    \label{fig:nemenyi_plot}
\end{figure}

% --- DISCUSSÃO ---
\section{Discussion}
Cypher's Edge outperforms the literature baseline in both accuracy and F1-score. Random Forest shows the highest discriminative capability, especially for KDA, damage, and economy-related metrics.

% --- CONCLUSÃO ---
\section{Conclusion}
Cypher's Edge demonstrates strong predictive potential and provides a fully reproducible framework. Dataset expansion allows automatic regeneration of metrics and figures via LaTeX, enabling continuous scientific validation.

% --- REFERÊNCIAS ---
\bibliographystyle{IEEEtran}
\begin{thebibliography}{00}
\bibitem{pan2019}
Pan, et al. “Beyond Accuracy: A Statistical Validation of Feature Engineering Methods for League of Legends Match Prediction,” 2019.
\end{thebibliography}

\end{document}
